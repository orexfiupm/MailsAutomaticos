% Created 2018-06-13 Wed 16:59
\documentclass[11pt]{article}
\usepackage[utf8]{inputenc}
\usepackage[T1]{fontenc}
\usepackage{fixltx2e}
\usepackage{graphicx}
\usepackage{longtable}
\usepackage{float}
\usepackage{wrapfig}
\usepackage{rotating}
\usepackage[normalem]{ulem}
\usepackage{amsmath}
\usepackage{textcomp}
\usepackage{marvosym}
\usepackage{wasysym}
\usepackage{amssymb}
\usepackage{hyperref}
\tolerance=1000
\author{Luciano Garcia Giordano}
\date{\textit{<2018-06-09 Sat>}}
\title{Envío semiautomático de correos - Practicum}
\hypersetup{
  pdfkeywords={},
  pdfsubject={},
  pdfcreator={Emacs 24.5.1 (Org mode 8.2.10)}}
\begin{document}

\maketitle
\tableofcontents


\section{Estructura de directorios}
\label{sec-1}
MailAutomaticos
\begin{itemize}
\item envio
\begin{itemize}
\item bdalumnos.xls
\item envio.conf
\end{itemize}
\item programas
\begin{itemize}
\item envio$_{\text{a}}$$_{\text{Alumno}}$.bat
\item envio$_{\text{a}}$$_{\text{TP}}$.bat
\item envio$_{\text{a}}$$_{\text{TA}}$.bat
\item envioCorreoaAlumno.jar
\item envioCorreoaTA.jar
\item envioCoreoaTP.jar
\end{itemize}
\item leeme
\item documentacion
\item configuraciones (.conf)
\end{itemize}

\section{Ejecución}
\label{sec-2}
\begin{enumerate}
\item Asegurarse de que envio/bdalumnos.xls es correcto
\begin{itemize}
\item es la versión deseada
\item tiene extensión xls
\end{itemize}
\item Asegurarse de que envio/envioCorreo.conf es correcto
\begin{itemize}
\item es adecuado al público al que se envía (a alumno, a TP, a TA)
\item los ficheros adjuntos, si hay son ficheros existentes
\end{itemize}
\item Ejecutar mediante programas/ el programa deseado. Pinchar con doble click en el nombre correspondiente al tipo de envío, pero en el fichero con extensión .bat
\begin{itemize}
\item Para cada envío se abre Thunderbird con el correo a solo un destinatario. Si hay algo mal, cambiar y enviar o cerrar la ventana sin guardar.
\item Pulsar la tecla Enter/Intro/Nueva línea para que se genere el siguiente correo
\end{itemize}
\end{enumerate}

\section{Columnas en la hoja de cálculo}
\label{sec-3}
Nota: el orden en la hoja de cálculo no importa, pero todas las columnas aquí listadas tienen que existir (aunque algunas casillas pueden estar vacías)
\begin{itemize}
\item APELLIDOS: Apellidos del alumno
\item NOMBRE: nombre del alumno
\item mailAlumno: dirección de correo electrónico del alumno
\item EMPRESA: empresa o laboratorio en que se desarrolla el Practicum
\item TA: nombre del Tutor Académico
\item mailTA: dirección de correo electrónico del Tutor Académico
\item TP: nombre del Tutor Profesional
\item mailTP: dirección de correo electrónico del Tutor Profesional
\item telTP: número de teléfono del Tutor Profesional
\end{itemize}

\section{Fichero de configuración}
\label{sec-4}
Nombre del fichero: envioCorreo.conf

Localización:
\begin{itemize}
\item MailAutomaticos/ para archivo de ficheros de configuración
\item MailAutomaticos/envio/ para el fichero a ser utilizado inmediatamente
\end{itemize}
\subsection{Estructura del fichero}
\label{sec-4-1}
ASUNTO: \{asunto deseado para el correo\}

ADJUNTOS: \{ficheros a ser adjuntados separados por comas\}

TEXTO: \{texto que se quiera enviar, con patrones que se explican en la sección siguiente\}

\subsubsection{Ejemplo de correo a algún tutor:}
\label{sec-4-1-1}

ASUNTO: Correo inicio al TA para avisar sus alumnos

ADJUNTOS: C:$\backslash$\ldots{}\ldots{}

TEXTO: Estimado ||TA||,
Se envía \ldots{}\ldots{}\ldots{}\ldots{}\ldots{}\ldots{}\ldots{}
Los alumnos tutorados son los siguientes: ||SECUENCIADEALUMNOS[EMPRESA,APELLIDOS,NOMBRE]||

\subsubsection{Ejemplo de correo a algún alumno:}
\label{sec-4-1-2}

ASUNTO: Correo inicio al alumno sobre su tutor

ADJUNTOS: C:$\backslash$\ldots{}\ldots{}..
TEXTO: Estimado ||NOMBRE|| ||APELLIDOS||,
Se envía \ldots{}\ldots{}\ldots{}\ldots{}\ldots{}\ldots{}\ldots{}
Su TA es ||TA||
Su TP es ||TP||


\subsection{Patrones}
\label{sec-4-2}
Se aplican directamente en la parte de texto del fichero de configuración para insertar datos de los alumnos o tutores que se sacan de la hoja de cálculo indicada.
\subsubsection{Nombre/email/teléfono}
\label{sec-4-2-1}
Se utiliza ||\{nombre de la característica\}||. Siempre sustituye por un único valor.

Ejemplo: ||TA||, ||NOMBRE||, ||mailAlumno||, \ldots{}
\subsubsection{Secuencia de alumnos}
\label{sec-4-2-2}
Se usa ||SECUENCIADEALUMNOS[\{características separadas por comas\}]||. Sirve para incluir listados de alumnos, por ejemplo en correos direccionados a los tutores académicos o profesionales.

Ejemplo: ||SECUENCIADEALUMNOS[EMPRESA,APELLIDOS,NOMBRE]||

Nota: Entre APELLIDOS y NOMBRE pueden darse dos casos para que el correo quede más bien escrito:
\begin{itemize}
\item NOMBRE después que APELLIDOS: se insertará una coma entre los dos
\item APELLIDOS después que NOMBRE: no se inserta coma. Solo un espacio
\end{itemize}
\subsection{Cosas a tener en cuenta:}
\label{sec-4-3}
La utilización de los patrones depende del programa que se ejecutará (y por tanto de la tarea a ser realizada). Si enviamos correos a tutores académicos o profesionales, los alumnos vendrán como listas, y por tanto hay que usar el patrón de secuencia para obtener su información correctamente. Si enviamos a un alumno, sus datos ya son individuales y no se utiliza el patrón de secuencia.

\section{Diagnóstico de problemas}
\label{sec-5}
\subsection{Correo aparece vacío}
\label{sec-5-1}
Comprobar que la ruta a algún posible fichero adjunto es correcta.
\subsection{No abre la ventana de ejecución}
\label{sec-5-2}
Comprobar en el fichero .bat que está siendo ejecutado si la versión de Java es la correcta.
% Emacs 24.5.1 (Org mode 8.2.10)
\end{document}
